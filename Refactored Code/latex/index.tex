\hypertarget{index_intro}{}\section{Introduction}\label{index_intro}
The objective of this code is to run convex hull algorithms and understand and analuyze their execution times. In addition, we have plotted the working of the program as an mp4 file to help visualize the working of each. Here we have made use of three algorithms\+:
\begin{DoxyItemize}
\item Graham Scan
\item Jarvis March
\item Kirk-\/\+Patrick Sidele Algorithm 
\end{DoxyItemize}\hypertarget{index_install}{}\section{Installation}\label{index_install}
There are two folders
\begin{DoxyItemize}
\item Refactored Code
\item Vizard\+\_\+\+Vizon
\end{DoxyItemize}

The first contains the optimized code for testing the run time of each the second contains the additional code for visualization \hypertarget{index_step1}{}\subsection{Testing the performance\+:}\label{index_step1}
Navigate to the Refactored code folder.~\newline
 To test the performance run the command-\/~\newline
 {\bfseries bash script}~\newline
 Take user input from the terminal. The performace for each will be displayed on the terminal output.~\newline
 {\bfseries Note\+: This currently supports up to 10$^\wedge$7 data points} \hypertarget{index_step2}{}\subsection{Testing the Visualization\+:}\label{index_step2}
Navigate to the Vizard\+\_\+\+Vizon folder.~\newline
 To test the performance run the command-\/~\newline
 {\bfseries bash script}~\newline
 Take user input from the terminal.~\newline
 After the execution is complete, there will be an animated plot of the working depending on the option that has been taken.~\newline
 In addition, there will be gifs generated, so that we do not have to run the algorithm one more time.~\newline
 \hypertarget{index_step3}{}\subsection{Conclusions\+:}\label{index_step3}
Jarvis march performs better on an average, however, the difference between them is not significant, in the sense that they both run in the same order of time.~\newline
 However, Kirkpartik-\/\+Sidel does not seem to scale up to that factor due to the fact that the use of vectors is not efficient due to the large number of memory operations. 